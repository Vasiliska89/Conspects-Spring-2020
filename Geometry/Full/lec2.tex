  \section{Замкнутые множества.}
\underline{\textit{$\circ$ Определение:}}\\
$F$ --- \emph{Замкнуто} $\Longleftrightarrow X\backslash F$ --- открыто.
\subsection{Теорема о множестве замкнутых множеств}
$ \mathcal{F} $ --- множество замкнутых множеств.
\\Тогда:
\begin{enumerate}
\item $\varnothing ,~X\in \mathcal{F}$
\item $F,~G\in \mathcal{F} \Longrightarrow F\cup G\in \mathcal{F}$
\item $F_{\alpha}\in \mathcal{F} \Longrightarrow \underset{\alpha\in I}{\cap}F_{\alpha}\in \mathcal{F}$
\end{enumerate}
 \textbf{Доказательство:}
 \begin{enumerate}
 	\item очевидно: $(X\backslash X = \varnothing~~;~~X\backslash \varnothing = X)$
 	\item $F\cup G$ --- замкнуто $\Longleftrightarrow X\backslash (F\cup G)$ ---открыто.\\
 	Но $X\backslash (F\cup G) = (X\backslash F)\cap(X\backslash G)$ -- открыто как пересечение двух открытых.
 	\item $X\backslash(\underset{\alpha\in I}{\cap}F_{\alpha}) = \underset{\alpha\in I}{\cup}(X\backslash F_{\alpha})$ --- рассуждение аналогично (2.)

\end{enumerate}
\subsection{Теорема о соответствующей топологии}
  Пусть $ \mathcal{F} \subseteq 2^{X}$  такое, что:
\begin{enumerate}
\item $\varnothing ,~X\in \mathcal{F}$
\item $F,~G\in \mathcal{F} \Longrightarrow F\cup G\in \mathcal{F}$
\item $F_{\alpha}\in \mathcal{F} \Longrightarrow \underset{\alpha\in I}{\cap}F_{\alpha}\in \mathcal{F}$
\end{enumerate}
$~~~~~~$Тогда существует единственная топология $\Omega$ такая, что $\mathcal{F}$ - множество замкнутых множеств.
\\
\textbf{Доказательство:}
 \begin{enumerate}
\item Докажем единственность:\\
Если такая топология $\Omega$ существует, то $\Omega = \lbrace X\backslash F|~F\in \mathcal{F}\rbrace$.
\\ Действительно, каждое такое множество должно входить в $\Omega$, и ни одно другое в неё входить не может. Таким образом, $\Omega$ - единственная возможная топология по построению.
\item Теперь докажем, что построенная $\Omega$ действительно является топологией. Для этого проверим 3 необходимых свойства из определения.
\begin{enumerate}
\item $\varnothing,~X \in\Omega$ -- очевидно.
\item $U = X\backslash F,~~V=X\backslash G; ~~F,~G\in \mathcal{F}$
\\$U\cap V = (X\backslash F)\cap (X\backslash G)=X\backslash (F\cup G) \Longrightarrow U\cap V\in\Omega.$ -- здесь мы пользуемся свойством (2).
\item $U_{\alpha} = X\backslash F_{\alpha};~~~F_{\alpha}\in \mathcal{F}$
\\$\underset{\alpha\in I}{\cup}U_{\alpha} = \underset{\alpha\in I}{\cup}X\backslash F_{\alpha} = X\backslash\underset{\alpha\in I}{\cap}F_{\alpha}$ -- здесь мы пользуемся свойством (3).
\end{enumerate} 
\end{enumerate}
\newpage
\section{Внутренность, замыкание и граница.}
\underline{\textit{$\circ$ Определения:}}\\
$A\subseteq X$
\begin{enumerate}
\item $x$ --- \emph{Внутренняя точка А}, если $\exists ~U_{x}\in\Omega:~U_{x}\subseteq A$ , т.е. точка входит в А с некоторой окрестностью. Заметим, что условие можно переписать как $(U_{x}\cap(X\backslash A))=\varnothing$
\item $x$ --- \emph{Внешняя точка А}, если $\exists ~U_{x}\in\Omega:~U_{x}\subseteq X\backslash A$ , т.е. точка входит в дополнение А с некоторой окрестностью. Иначе говоря, $U_{x}\cap A=\varnothing$.
\item $x$ --- \emph{Граничная точка А}, если $\forall ~U_{x}\in\Omega:~(U_{x}\cap A\neq\varnothing) ~\& ~ (U_{x}\cap(X\backslash A)\neq\varnothing)$ , т.е. любая окрестность пересекает и А, и дополнение А.
\\ \underline{\emph{$\bullet$ Упражнение: }} Доказать, что в случае метрических пространств определения останутся эквивалентными при замене окрестностей на шаровые.
\item $x$ --- \emph{Точка прикосновения А}, если $\forall~U_{x}~~U_{x}\cap A\neq \varnothing$
\item \emph{Внутренность} $Int(A)$ --- наибольшее по включению открытое множество, содержащееся в А.
\\ \underline{\emph{Примечание:}} Также используется обозначение $\overset{\circ}{A}$.
\item \emph{Замыкание} $Cl(A)$ --- наименьшее по включению замкнутое множество, содержащее А.
\\ \underline{\emph{Примечание:}} Также используется обозначение $\overline{A}$ и $cl(A)$.
\item \emph{Граница} $\partial A$ --- множество граничных точек А. 
\\ \underline{\emph{Примечание:}} Также используется обозначение $Fr(A)$. Если необходимо подчеркнуть, к какому всеобъемлещему множеству относится граница, пишут $\partial_{X}A$.
\end{enumerate}
\subsection{Существование внутренности и замыкания}
\begin{enumerate}
\item $Int(A)$ существует и $Int(A)\overset{1}{=}\underset{\underset{U\in\Omega}{U\subseteq A,}}{\cup}U\overset{2}{=}\lbrace ~x~ |~~x$-- внутр.$\rbrace$
\\
\textbf{Доказательство:}
\\ \underline{1.} Первое равенство справедливо, т.к. объединение открытых множеств открыто, содержится в А и любое открытое подмножество А лежит в нем по определению.
\\ \underline{2.} Покажем, что второе множество содержится в первом и наоборот.
\begin{enumerate}
\item Пусть $x$ --- внутренняя точка А. Тогда существует $U_{x}\subseteq A$. Но $U_{x}\subseteq \underset{\underset{U\in\Omega}{U\subseteq A,}}{\cup}U$, значит $x\in \underset{\underset{U\in\Omega}{U\subseteq A,}}{\cup}U$, то есть $\lbrace ~x~ |~~x$-- внутр.$\rbrace\subseteq \underset{\underset{U\in\Omega}{U\subseteq A,}}{\cup}U$
\item Пусть точка $x$ лежит в нашем объединении. Значит существует открытое множество, в котором она лежит. Но тогда x - внутренняя точка А, то есть любая точка из объединения является внутренней, а значит $\lbrace ~x~ |~~x$-- внутр.$\rbrace\supseteq \underset{\underset{U\in\Omega}{U\subseteq A,}}{\cup}U$.
\\Значит множества действительно равны. Заметим, что иногда последняя часть равенства используется в качестве определения.
\end{enumerate}
\item $Cl(A)$ существует и $Cl(A)\overset{1}{=}\underset{\underset{F\in\mathcal{F}}{ F\supseteq A,}}{\cap}F\overset{2}{=}\lbrace x|~~x$---т. прикосн. A$\rbrace$.
\\ \textbf{Доказательство:}
\\$R:=\underset{\underset{F\in\mathcal{F}}{ F\supseteq A,}}{\cap}F\supseteq A$, R - замкнуто как пересечение замкнутых. Если замкнутое множество G содержит А, то G содержит и R, ведь G входит в пересечение. Значит R - действительно наименьшее (по включению) замкнутое множество, содержащее А.
\end{enumerate}
\subsection{Свойства замыкания, внутренности и границы.}
\begin{enumerate}
\item $X\backslash Cl(A) = Int(X\backslash A)$. - это очевидно, но давайте докажем:
\\ \textbf{Доказательство:}
\\ $X\backslash Cl(A) = X\backslash(\cap F)$\footnote{Здесь и далее индексы объединения/пересечения/суммирования пишутся только в первом употреблении и далее опускаются чтобы не загромождать текст. Как правило всё ясно из контекста и без них.}$=\cup(X\backslash F)=[U = X\backslash F$ -открыто$]=\underset{U\subseteq(X\backslash A)}{\cup U}=Int(X\backslash A)$ .
\item $X\backslash Cl(A)$ - множество всех внешних точек А. Следует напрямую из пункта 1.
\item $X\backslash Int(A) = Cl(X\backslash A)$. Доказывается абсолютно аналогично пункту 1.
\end{enumerate}
Выпишем ещё несколько свойств, оставив часть из них читателю в качестве упражнения.
\begin{enumerate}
\item 
\begin{enumerate}
\item A - откр. $\Longleftrightarrow ~A = Int(A)$
\item A - замкнуто $\Longleftrightarrow A = Cl(A)$
\end{enumerate}
\item 
\begin{enumerate}
\item $A\subseteq B \Longrightarrow Int(A)\subseteq Int(B)$
\item $A\subseteq B \Longrightarrow Cl(A)\subseteq Cl(A)$
\end{enumerate}
\item
\begin{enumerate}
\item $Int(A\cap B) = Int(A)\cap Int(B)$
\item $Cl(A\cup B) = Cl(A)\cup Cl(B)$
\end{enumerate}
\item
\begin{enumerate}
\item $Int(A\cup B) \supseteq Int(A)\cup Int(B)$
\item $Cl(A\cap B) \subseteq Cl(A)\cap Cl(B)$
\end{enumerate}
\end{enumerate}
Приведем лишь доказательства пунктов (а), пункт (b) же выводится либо из (а), либо аналогично (а).
\\ Итак, начнем.
\begin{enumerate}
\item Как и всегда в таких теоремах, разобьем утверждение на 2 и докажем их по отдельности.
\\ $\underline{\Leftarrow}$: На самом деле очевидно, т.к. $A = Int(A) = \cup U_{x}$ --- открыто как объединение открытых.
\\ $\underline{\Rightarrow}$: А --- открыто, $A\subseteq A$. Значит А входит в объединение открытых подмножеств, содержащихся в А, откуда очевидно $A = Int(A)$.
\item Воспользуемся другим определением внутренности: $Int(A) = \lbrace~x|~~x$ - внутр. в А$\rbrace$.
\\ Заметим, что если x - внутр. в А, $A\subseteq B$, то x - внутр. в B ($U_{x}\subseteq A\subseteq B$). Откуда $Int(A) = \lbrace~x|~~x$ - внутр. в А$\rbrace\subseteq\lbrace~x|~~x$ - внутр. в B$\rbrace=Int(B)$.
\item Существует по меньшей мере 2 доказательства данного факта. Мы приведем лишь один, оставив второй на совесть читателя.
\\ Как и всегда в таких ситуациях, разобьем утверждение на 2 и докажем их по отдельности.
\\ $\underline{\subseteq}$: $A\cap B\subseteq A\Longrightarrow Int(A\cap B)\subseteq Int(A)$ (по пункту (2)).
\\Аналогично,  $Int(A\cap B)\subseteq Int(B)$, откуда $Int(A\cap B)\subseteq Int(A)\cap Int(B)$.
\\ $\underline{\supseteq}$: \begin{equation*}
\begin{cases}
    & A\supseteq Int(A)\\
    & B\supseteq Int(B)
 \end{cases}~~~~~~~|\Longrightarrow A\cap B \supseteq Int(A)\cap Int(B)\Longrightarrow Int(A\cap B)\supseteq Int(A)\cap Int(B)
\end{equation*}
(здесь используются сразу пункты 1 и 2)
\item Пусть $x$ --- внутренняя точка А. Тогда она внутренняя и для $A\cup B$, значит $Int(A)\subseteq Int(A\cup B)$. Аналогично для $Int(B)$.
\\ Тогда $Int(A)\cup Int(B)\subseteq Int(A\cup B)\cup Int(A\cup B)=Int(A\cup B)$.
 \begin{flushright}
$QED.$\footnote{Пункты 1-2,4 созданы вашим покорным слугой. Лекционным считать только пункт 3.}
\end{flushright}
\end{enumerate}
\section{База топологии.}
\underline{\textit{$\circ$ Определение:}}\\
$(X,~\Omega)$ - топологическое пространство.
$\Sigma\subseteq\Omega$ --- \emph{База топологии}, если любой элемент из $\Omega$ представим в виде объединения некоторых элементов из $\Sigma$.
\subsection{Критерий базы топологии $\Omega$}
Оказывается, что определение базы $\Sigma$ для топологии $\Omega$ эквивалентно совокупности следующих двух условий:
\begin{enumerate}
\item $\forall U\in\Omega~\forall x\in U \exists S\in \Sigma:~x\in S\subseteq U$
\item $\forall S\in \Sigma:~~ S$ --- откр.
\end{enumerate}
\textbf{Доказательство:}\\
$\underline{\Rightarrow}$: 
\begin{enumerate}
\item По условию, $\forall U\in \Omega~\exists I:~\underset{\alpha\in I}{\cup}S_{\alpha}\in \Sigma:~~U=\underset{\alpha\in I}{\cup}S_{\alpha}$
\\ Тогда $x\in \underset{\alpha\in I}{\cup}S_{\alpha}\Longrightarrow \exists\alpha_{0}\in I:~x\in S_{\alpha_{0}}$, но $S_{\alpha_{0}}\subseteq\underset{\alpha\in I}{\cup}S_{\alpha}=U$. $S = S_{\alpha_{0}}$ - подходит.
\item очевидно из определения.
\end{enumerate}
$\underline{\Leftarrow}$: \begin{enumerate}
\item[]По усновию, $\forall x\in U \exists S_{x}\in\Sigma:~~x\in S_{x}\subseteq U$
\\Но тогда $U\subseteq\underset{\alpha\in I}{\cup}S_{\alpha}\subseteq U~~\Longleftrightarrow~\underset{\alpha\in I}{\cup}S_{\alpha}=U$.
\end{enumerate}
 \underline{\emph{$\bullet$ Упражнение: }} Докажите, что шары - база метрической топологии.
\subsection{Критерий базы некоторой топологии}
Оказывается, что $\Sigma$ --- база некоторой топологии тогда и только тогда, когда выполняются следующие два условия:
\begin{enumerate}
\item $\forall x\in X ~\exists S_{x}:~~x\in S_{x}$
\item $\forall S_{1,2}\in \Sigma: ~~\forall x\in S_{1}\cap S_{2}~~ \exists S_{x}\in\Sigma:~x\in S_{x}\subseteq S_{1}\cap S_{2}$
\\ Причем заданная топология единственна.
\end{enumerate}
 \textbf{Доказательство:}
\begin{enumerate}
\item[$\underline{\Rightarrow}$]:
\begin{enumerate}
\item[1.] $\forall x\in X\in \Omega\Longrightarrow~\exists I:~ S_{\alpha}\in \Sigma ~~\&~~ X = \underset{\alpha\in I}{\cup S_{\alpha}} $
\\$|\Rightarrow~~ \forall x\in X \exists \alpha:~~x\in S_{\alpha}$.
\item[2.] $S_{1}\cap S_{2}$  - откр. $\Longrightarrow~~\exists I: S_{\alpha}\in\Sigma;$\\
$S_{1}\cap S_{2}=\underset{\alpha\in I}{\cup S_{\alpha}}~\Longrightarrow \exists\alpha:~x\in S_{\alpha}\subseteq S_{1}\cap S_{2}$. 
\end{enumerate}
\item[$\underline{\Leftarrow}$]:
\\ Если топология существует, то $\Omega = {\underset{\alpha\in I}{\cap}S_{\alpha}|~~S_{\alpha}\in\Sigma}$
\\ Если это топология, то $\Sigma$ --- её база. Проверим, что $\Omega$ --- действительно топология.
\begin{enumerate}
\item[1)] $\varnothing$ --- "пустое объединение"($|I|=0$).
\\X есть по свойству 1. Например, можно взять объединение всех S из $\Omega$.
\item[2)]$U, V\in\Omega,~\Longleftrightarrow ~ U=\cup S_{\alpha}, V = \cup S_{\beta}$.
\\ $U\cap V =(\cup S_{\alpha})\cap(\cup S_{\beta}) = \cup(S_{\alpha}\cap S_{\beta})\fbox{=} $
\\ $\forall x\in S_{\alpha}\cap S_{\beta}~~\exists S_{x}: x\in S_{x}\subseteq S_{\alpha}\cap S_{\beta}$      (свойство 2)
\\ $S_{\alpha}\cap S_{\beta}\subseteq\underset{x\in S_{\alpha}\cap S_{\beta}}{\cup}S_{x}\subseteq S_{\alpha}\cap S_{\beta}$
\\ $\fbox{=}\underset{\alpha,\beta}{\cup}(\underset{x\in S_{\alpha}\cap S_{\beta}}{\cup}S_{x})$ - лежит в $\Omega$. Объясним это в пункте 3, попутно доказав его.
\item[3)] $\underset{\beta\in I}{\cup}S_{\beta} = \underset{\beta\in I}{\cup}(\underset{\alpha\in I_{b}}{\cup}S_{\alpha})$ - это объединение объединений множеств из базы. Поскольку объединение ассоциативно, это просто объединение множеств из базы. А значит, оно лежит в $\Omega$.
\begin{flushright}
$QED.$
\end{flushright}
\end{enumerate}
\end{enumerate}
\subsubsection{Пример}
$X, Y$ -- топологические пространства
\\ $$\lbrace U\times V |~~ U \text{-- откр. в X}~,~ V \text{-- откр. в Y}\rbrace \text{ --- База топологии.}$$
\\Доказывается непосредственно.
