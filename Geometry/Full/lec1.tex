    
    

\section{Метрическое пространство. Метрика.}
\underline{\textit{$\circ$ Определение:}}\\

  $d: X \times X\rightarrow \mathbb{R}$ --- \emph{\textit{метрика}}, если:
 \begin{enumerate}
\item $~d\left( x, y\right)\geq 0;~~ d\left( x, y\right) = 0\Longleftrightarrow x=y ,$
\item $~d\left( x, y\right)=d\left( y, x\right) ,$
\item $~d\left( x, y\right)+d\left( y, z\right)\geq d(x, z) .$

\end{enumerate}
 (X, d) --- \textit{метрическое пространство.}
 \\
 \subsection{Примеры метрик.}

 \begin{enumerate}
\item V - векторное пространство, $<\cdot , \cdot >$ - скалярное произведение.
 \\$ ~~~~~~\sphericalangle ~~d(\vec{v_{1}}, \vec{v_{2}}) = \sqrt{<\vec{v_{1}} - \vec{v_{2}},\vec{v_{1}} - \vec{v_{2}}> } = |\vec{v_{1}} - \vec{v_{2}}|$
 \item (E, V) -- аффинное пространство, скалярное произведение на V.
 \item \emph{$\circ$ Определение:}\\
 \emph{Норма }$||\cdot||: V\rightarrow  \mathbb{R}$
 \subitem $1. ~~||\vec{v}||\geq 0 ; ~~||\vec{v}|| = 0 \Longleftrightarrow \vec{v} = \vec{0} $
 \subitem $ 2. ~~||\lambda\vec{v}|| = |\lambda|*||\vec{v}|| $
 \subitem $ 3.~~||\vec{v_{1}} + \vec{v_{2}} ||\leq || \vec{v_{1}} || + ||\vec{v_{2}}||$
 \\Тогда $d(\vec{v_{1}}, \vec{v_{2}}) = ||\vec{v_{1}} - \vec{v_{2}}||$ - метрика.
 \\ \\3.1 $V = \mathbb{R}^{n} ,~~~ ||x||_{p} = (\sum |x_{i}|^{p})^{1/p},~~~ p\geq 1$
 \subsubitem $~~~~~~~~~~||x||_{\inf}=\underset{1\leq i\leq n}{max}(|x_{i}|)$
 \item $S^{2}$ --- сфера, 2 метрики:
 \subsubitem $1. \mathbb{R}^{3}\rightarrow \mathbb{R}$ (Стандартная метрика),
 \subsubitem 2. Угловая метрика.
\end{enumerate}
\end{spacing}
\newpage
\section{Открытые множества. Топологические пространства.}
\subsection{Открытые множества. Свойства.}
\underline{\textit{$\circ$ Определения:}}\\
\begin{enumerate}
\item $R>0,~B(x_{0}, R)\overset{def}{=}\lbrace x\in X|~~d(x_{0},x)<R\rbrace$ ---\emph{открытый шар радиуса R с центром $x_{0}$}.
\item $S(x_{0}, R)\overset{def}{=}\lbrace x\in X|~~d(x_{0}, x)=R\rbrace$ ---\emph{сфера}.
\item $(X, d)$ --метрическое пространство, $U\subseteq X$
\\$U$ -- \emph{открыто}, если:
\subitem $\forall x\in U \exists \varepsilon >0 :~~B(x, \varepsilon)\subseteq U.$
\end{enumerate}

\subsubsection{Теорема об открытости шара.}
Открытый шар открыт.
\begin{spacing}{1.5}
 \textbf{Доказательство:}\\
Рассмотрим $ x\in B(x_{0}, R).  \\
  \varepsilon :=R-d(x_{0}, x)>0 $
 (т.к. $x\in B(x_{0}, R)\Rightarrow d(x_{0}, x)<R)$
 \\Пусть $y\in B(x, \varepsilon)\Rightarrow d(x, y)<\varepsilon = R-d(x_{0}, x) $
 \\$d(x_{0}, y)\leq d(x_{0}, x)+d(x, y)<R\Rightarrow y\in B(x_{0}, R)\Rightarrow B(x, \varepsilon)\subseteq B(x_{0}, R). $
\begin{flushright}
$QED.$
\end{flushright}
\subsubsection{Теорема о свойствах $\Omega$}
$\Omega \overset{def}{=} \lbrace U\subseteq X |~~ U$--открыто$\rbrace;~~X$--метрическое пространство.
\\Тогда:
\begin{enumerate}
\item $\varnothing ,~X\in \Omega$
\item $U,~V\in \Omega \Rightarrow U\cap V\in \Omega$
\item $U_{\alpha}\in \Omega \Rightarrow \underset{\alpha\in I}{\cup}U_{\alpha}\in \Omega$
\end{enumerate}
 \textbf{Доказательство:}
 \begin{enumerate}
 	\item 
 	$1)~\varnothing$---очевидно.
 	$2)~$Рассмотреть для каждого элемента $\varepsilon=1$ (или любой другой, далее очевидно.)
 	\item Рассмотрим $ y\in U\cap V$
 	\\$y\in U \Rightarrow \exists \varepsilon_{1}>0:\forall x\in X~~ d(x, y)<\varepsilon_{1}\Rightarrow x\in U$
 	\\$y\in V \Rightarrow \exists \varepsilon_{2}>0:\forall x\in X~~ d(x, y)<\varepsilon_{2}\Rightarrow x\in V$
 	\\$\sphericalangle \varepsilon:=min(\varepsilon_{1}, \varepsilon_{2}).$
 	\\Далее очевидно.
 	\item $x\in \underset{\alpha\in I}{\cup}U_{\alpha}\Rightarrow \exists i\in I: x\in U_{i}\Rightarrow\exists\varepsilon : (\forall y\in X~~ d(x, y)<\varepsilon\Rightarrow y\in U_{i}\Rightarrow y\in \underset{\alpha\in I}{\cup}U_{\alpha}) $
 	\begin{flushright}
$QED.$
\end{flushright}
\end{enumerate}
\subsection{Топологическое пространство. Топология.}
 \underline{\textit{$\circ$ Определение:}}\\
 $(X, \Omega)$--- \emph{топологическое пространство}, если:
 \begin{enumerate}
 \item $\varnothing,~X\in \Omega$
 \item $U, V\in \Omega \Rightarrow U\cap V\in \Omega$
 \item $U_{\alpha}\in \Omega \Rightarrow \underset{\alpha\in I}{\cup}U_{\alpha}\in \Omega$
 
 \end{enumerate}
  $\Omega$ -- \emph{топология}, элементы $\Omega$ -- \emph{открытые.}
\subsubsection{Примеры топологических пространств.}
\begin{enumerate}
\item $\Omega=2^{X}$ -- \emph{дискретная топология.}\\
\underline{\emph{$\bullet$ Упражнение: }} Доказать, что \emph{дискретной метрике} $d(x, y) = \begin{cases}
  0,  &  x=y  \\
  1, &  x\neq y 
\end{cases}$ соответствует дискретная топология.
\item $\Omega=\lbrace\varnothing,~X\rbrace$ - \emph{антидискретная топология}
\item $X=\mathbb{R};~~ \Omega=\lbrace (a, +\inf)|~~a\in \mathbb{R} \rbrace\cup\lbrace\varnothing,~ \mathbb{R}\rbrace $ - \emph{стрелка}
\end{enumerate}
\subsection{Окрестности. Теорема о дополнении подмножества.}
\underline{\textit{$\circ$ Определения:}}\\
$x_{0}\in X.$
\begin{enumerate}
\item \emph{Окрестность точки $x_{0}$} --- произвольное открытое множество, содержащее $x_{0}.$
\item \emph{$\varepsilon$--окрестность (.) $x_{0}$}--$B(x_{0},\varepsilon).$ Определено только для метрических пространств.
\end{enumerate}
\subsubsection{Теорема о дополнении подмножества}
$F\subset X$;  $X$ --- метрическое пространство
\begin{enumerate}
\item $(\forall x~~ \forall\varepsilon~~(B(x,\varepsilon)\backslash\lbrace x\rbrace\cap F\neq\varnothing)\Rightarrow x\in F)\Longleftrightarrow X\backslash F$--открыто.
\item $X$--- топологическое пространство.\\
$((\forall U_{x}$---откр. $U_{x}\cap F\backslash\lbrace x\rbrace\neq \varnothing) \Rightarrow x\in F)\Longleftrightarrow X\backslash F$ --- открыто.
\end{enumerate}
\textbf{Доказательство:}
 \begin{enumerate}
 \item Если $X$--- метрическое пространство, то $1.\Longleftrightarrow 2.~:$
 \\ $ 1.\Leftarrow 2.:$ -- очевидно
 ; $ 1.\Rightarrow 2.:$
 \\Рассмотрим $ U_{x}$ --откр. $\Rightarrow \exists\varepsilon: B(x, \varepsilon)\subseteq U_{x}\Rightarrow((B(x, \varepsilon)\backslash\lbrace x\rbrace ) \cap F\subset U_{x}\backslash\lbrace x\rbrace\cap F\Rightarrow U_{x}\backslash\lbrace x\rbrace\cap F\neq \varnothing.$
 \begin{flushright}
$QED.$
\end{flushright}
 \end{enumerate}
Достаточно доказать (2.)
 \begin{enumerate}
 \item $\underline{\Rightarrow}$, т.е. $x\notin F \Rightarrow \exists U_{x}: U_{x}\backslash \lbrace x\rbrace\cap F=\varnothing\Rightarrow U_{x}\cap F=\varnothing\Rightarrow U_{x}\subseteq X\backslash F\Rightarrow \underset{x\in X\backslash F}{\cup}U_{x}\subseteq X\backslash F  $
 ; ''$\supseteq$''- тоже(оч.), $\Rightarrow\underset{x\in X\backslash F}{\cup}U_{x}= X\backslash F$, но $\cup U$ - откр.
 \begin{flushright}
$QED.$
\end{flushright}
\item $\underline{\Leftarrow} :$
\\ Пусть $X\backslash F$--откр. и $\forall U_{x}$--откр.   $(U_{x}\backslash\lbrace x\rbrace\cap F\neq\varnothing)$ 
\\Допустим, что $x\notin F~\Rightarrow ~ x\in (X\backslash F)$
\\Тогда $\varnothing \neq ((X\backslash F)\backslash\lbrace x\rbrace)\cap F = ((X\backslash F)\cap F)\backslash \lbrace x\rbrace = \varnothing.$ Противоречие.\footnote{Последняя часть доказательства является авторской. Лекционный вариант утерян бесследно.}
 \begin{flushright}
$QED.$
\end{flushright}
  \end{enumerate}