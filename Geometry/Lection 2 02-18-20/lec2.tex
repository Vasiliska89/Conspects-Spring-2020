\documentclass{article}
\usepackage{blindtext}
\usepackage[T1]{fontenc}
\usepackage[utf8]{inputenc}
\usepackage{amssymb}
\usepackage{setspace}
\usepackage{amsmath}
\usepackage{indentfirst}
\usepackage{mathtools}
\usepackage[margin=2cm]{geometry}
\usepackage[russian]{babel}
\title{Лекция 1.}
\author{Косовский Н.Н.}
\date{18 февраля 2020 г.}

\begin{document}
\begin{spacing}{1.5}
\maketitle
\newpage
     
    \tableofcontents
    \newpage

\section{Замкнутые множества.}
\underline{\textit{$\circ$ Определение:}}\\
$F$ --- \emph{Замкнуто} $\Longleftrightarrow X\backslash F$ --- открыто.
\subsection{Теорема о множестве замкнутых множеств}
$ \mathcal{F} $ --- множество замкнутых множеств.
\\Тогда:
\begin{enumerate}
\item $\emptyset ,~X\in \mathcal{F}$
\item $F,~G\in \mathcal{F} \Rightarrow F\cup G\in \mathcal{F}$
\item $F_{\alpha}\in \mathcal{F} \Rightarrow \underset{\alpha\in I}{\cap}F_{\alpha}\in \mathcal{F}$
\end{enumerate}
 \textbf{Доказательство:}
 \begin{enumerate}
 	\item очевидно: $(X\backslash X = \emptyset~~;~~X\backslash \emptyset = X)$
 	\item $F\cup G$ --- замкнуто $\Longleftrightarrow X\backslash (F\cup G)$ ---открыто.\\
 	Но $X\backslash (F\cup G) = (X\backslash F)\cap(X\backslash G)$ -- открыто как пересечение двух открытых.
 	\item $X\backslash(\underset{\alpha\in I}{\cap}F_{\alpha}) = \underset{\alpha\in I}{\cup}(X\backslash F_{\alpha})$ --- рассуждение аналогично (2.)

\end{enumerate}
\subsection{Теорема о соответствующей топологии}
  Пусть $ \mathcal{F} \subseteq 2^{X}$  такое, что:
\begin{enumerate}
\item $\emptyset ,~X\in \mathcal{F}$
\item $F,~G\in \mathcal{F} \Rightarrow F\cup G\in \mathcal{F}$
\item $F_{\alpha}\in \mathcal{F} \Rightarrow \underset{\alpha\in I}{\cap}F_{\alpha}\in \mathcal{F}$
\end{enumerate}
$~~~~~~$Тогда существует единственная топология $\Omega$ такая, что $\mathcal{F}$ - множество замкнутых множеств.
\\
\textbf{Доказательство:}
 \begin{enumerate}
\item Докажем единственность:\\
Если такая топология $\Omega$ существует, то $\Omega = \lbrace X\backslash F|~F\in \mathcal{F}$
\\ Действительно, каждое такое множество должно входить в $\Omega$, и ни одно другое в неё входить не может. Таким образом, $\Omega$ - единственная возможная топология по построению.
\item Теперь докажем, что построенная $\Omega$ действительно является топологией. Для этого проверим 3 необходимых свойства из определения.
\begin{enumerate}
\item $\emptyset,~X \in\Omega$ -- очевидно.
\item $U = X\backslash F,~~V=X\backslash G; ~~F,~G\in \mathcal{F}$
\\$U\cap V = (X\backslash F)\cap (X\backslash G)=X\backslash (F\cup G) \Rightarrow U\cap V\in\Omega.$ -- здесь мы пользуемся свойством (2).
\item $U_{\alpha} = X\backslash F_{\alpha};~~~F_{\alpha}\in \mathcal{F}$
\\$\underset{\alpha\in I}{\cup}U_{\alpha} = \underset{\alpha\in I}{\cup}X\backslash F_{\alpha} = X\backslash\underset{\alpha\in I}{\cap}F_{\alpha}$
\end{enumerate} 
\end{enumerate}




\end{spacing}


\end{document}